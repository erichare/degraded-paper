\documentclass[12pt,]{article}
\usepackage{lmodern}
\usepackage{amssymb,amsmath}
\usepackage{ifxetex,ifluatex}
\usepackage{fixltx2e} % provides \textsubscript
\ifnum 0\ifxetex 1\fi\ifluatex 1\fi=0 % if pdftex
  \usepackage[T1]{fontenc}
  \usepackage[utf8]{inputenc}
\else % if luatex or xelatex
  \ifxetex
    \usepackage{mathspec}
  \else
    \usepackage{fontspec}
  \fi
  \defaultfontfeatures{Ligatures=TeX,Scale=MatchLowercase}
\fi
% use upquote if available, for straight quotes in verbatim environments
\IfFileExists{upquote.sty}{\usepackage{upquote}}{}
% use microtype if available
\IfFileExists{microtype.sty}{%
\usepackage{microtype}
\UseMicrotypeSet[protrusion]{basicmath} % disable protrusion for tt fonts
}{}
\usepackage[margin=1in]{geometry}
\usepackage{hyperref}
\hypersetup{unicode=true,
            pdftitle={Algorithmic Approaches to Match Degraded Land Impressions},
            pdfauthor={Eric Hare, Heike Hofmann, Alicia Carriquiry},
            pdfborder={0 0 0},
            breaklinks=true}
\urlstyle{same}  % don't use monospace font for urls
\usepackage{longtable,booktabs}
\usepackage{graphicx,grffile}
\makeatletter
\def\maxwidth{\ifdim\Gin@nat@width>\linewidth\linewidth\else\Gin@nat@width\fi}
\def\maxheight{\ifdim\Gin@nat@height>\textheight\textheight\else\Gin@nat@height\fi}
\makeatother
% Scale images if necessary, so that they will not overflow the page
% margins by default, and it is still possible to overwrite the defaults
% using explicit options in \includegraphics[width, height, ...]{}
\setkeys{Gin}{width=\maxwidth,height=\maxheight,keepaspectratio}
\IfFileExists{parskip.sty}{%
\usepackage{parskip}
}{% else
\setlength{\parindent}{0pt}
\setlength{\parskip}{6pt plus 2pt minus 1pt}
}
\setlength{\emergencystretch}{3em}  % prevent overfull lines
\providecommand{\tightlist}{%
  \setlength{\itemsep}{0pt}\setlength{\parskip}{0pt}}
\setcounter{secnumdepth}{5}
% Redefines (sub)paragraphs to behave more like sections
\ifx\paragraph\undefined\else
\let\oldparagraph\paragraph
\renewcommand{\paragraph}[1]{\oldparagraph{#1}\mbox{}}
\fi
\ifx\subparagraph\undefined\else
\let\oldsubparagraph\subparagraph
\renewcommand{\subparagraph}[1]{\oldsubparagraph{#1}\mbox{}}
\fi

%%% Use protect on footnotes to avoid problems with footnotes in titles
\let\rmarkdownfootnote\footnote%
\def\footnote{\protect\rmarkdownfootnote}

%%% Change title format to be more compact
\usepackage{titling}

% Create subtitle command for use in maketitle
\newcommand{\subtitle}[1]{
  \posttitle{
    \begin{center}\large#1\end{center}
    }
}

\setlength{\droptitle}{-2em}
  \title{Algorithmic Approaches to Match Degraded Land Impressions}
  \pretitle{\vspace{\droptitle}\centering\huge}
  \posttitle{\par}
  \author{Eric Hare, Heike Hofmann, Alicia Carriquiry}
  \preauthor{\centering\large\emph}
  \postauthor{\par}
  \predate{\centering\large\emph}
  \postdate{\par}
  \date{10/05/2017}

\usepackage{float}

\usepackage{amsthm}
\newtheorem{theorem}{Theorem}[section]
\newtheorem{lemma}{Lemma}[section]
\theoremstyle{definition}
\newtheorem{definition}{Definition}[section]
\newtheorem{corollary}{Corollary}[section]
\newtheorem{proposition}{Proposition}[section]
\theoremstyle{definition}
\newtheorem{example}{Example}[section]
\theoremstyle{definition}
\newtheorem{exercise}{Exercise}[section]
\theoremstyle{remark}
\newtheorem*{remark}{Remark}
\newtheorem*{solution}{Solution}
\begin{document}
\maketitle

\textbf{Abstract}

Bullet matching is a process used to determine whether two bullets may
have been fired from the same gun barrel. Historically, this has been a
manual process performed by trained forensic examiners. Recent work
however has shown that it is possible to add statistical validity and
objectivity to the procedure. In this paper, we build upon the
algorithms explored in Automatic Matching of Bullet Lands (Hare,
Hofmann, and Carriquiry 2016) by formalizing and defining a set of
features, computed on pairs of bullet lands, which can be used in
machine learning models to assess the probability of a match. We then
use these features to perform an analysis of the two Hamby (Hamby,
Brundage, and Thorpe 2009) bullet sets (Set 252 and Set 44), to assess
the presence of microscope operator effects in scanning. We also take
some first steps to address the issue of degraded bullet lands, and
provide a range of degradation at which the matching algorithm still
performs well. Finally, we discuss generalizing land to land comparisons
to full bullet comparisons as would be used for this procedure in a
criminal justice situation.

\section{Background}\label{background}

Intense scrutiny has been focused on the process of bullet matching in
recent years (e.g., Giannelli 2011). Bullet matching, the process of
determining whether two bullets could have been fired from the same gun
barrel, has traditionally been performed without meaningful
determination of error rates or statistical assessments of uncertainty
(National Research Council 2009). There have been some attempts towards
developing mathematical and statistical approaches to bullet matching.
One such attempt was the definition of CMS, the Consecutively Matching
Striae (Biasotti 1959), with a cutoff of six to separate matches from
non-matches. Still, rigorous assessments of the applicability of such
cutoffs have not to this point been described (Advisors on Science and
Technology 2016).

Recently, several authors have addressed these well-known shortcomings.
Focusing on firing pin impressions and breech faces, Riva and Champod
(2014) have described an automated algorithm using 3D images that
enables comparison between pairs of exemplars. Other examples of work in
this and related areas include Petraco and Chan (2012), W. Chu et al.
(2011), T. Vorburger et al. (2011), and others. In our approach to this
problem, Automatic Matching of Bullet Lands, we used the Hamby 252 set
(Hamby, Brundage, and Thorpe 2009) to train and develop a random forest
in order to provide a matching probability for two bullet lands (Hare,
Hofmann, and Carriquiry 2016). While the algorithm had a very strong
performance on this set, some limitations were immediately clear. For
instance, performance was assessed only on this single set of 35 bullets
fired from a consecutively manufactured set of only ten known and 15
unknown gun barrels. Each of these bullets was part of controlled study,
and the full lands were available for matching. While there were some
data quality issues, this was still a near ideal test case for the
algorithm.

Real world applications of bullet matching often involve the recovery of
fragments of bullets from the crime scene (Council and others 2004).
Traditional features used in forensic examination work well for a full
land, but there has been less investigation into their performance in
the case of a fragmented land. For example, the CMS is naturally limited
by the portion of the land that can be recovered and varies across
manufacturers (W. Chu et al. 2011).

In this paper, we take steps to address these and other concerns.
Specifically, we begin by reviewing features from the literature,
computed on pairs of bullet lands, and presenting some of our own
features. We propose an approach to standardize the featues, to account
for the fact that only a portion of the land impression may be recovered
from the crime scene. With the standardized features, we tackle two
issues that were not addressed in (Hare, Hofmann, and Carriquiry 2016).
The first is the effect of the microscope operator on the resulting
images and consequent algorithm performance. The second issue has to do
with the robustness of the land matching algorithm in (Hare, Hofmann,
and Carriquiry 2016) relative to the degree of degradation of the
questioned land impression. Finally, we describe some of the initial
steps toward generalizing a matching algorithm based on land-to-land
comparisons, to one based on bullet-to-bullet comparisons, as would be
of interest in a real world application of these ideas.

\section{Feature Standardization}\label{feature-standardization}

To start, we introduce a standardized version of each of the features
used in the matching routine proposed by (Hare, Hofmann, and Carriquiry
2016). These features are computed on \emph{aligned pairs of bullet land
impressions} rather than on individual lands. This enables us, for
instance, to compute the number of matching striae between two lands. We
generalize the definitions of these features to account for the
possibility that we may be handling degraded bullet lands, where only
fragments can be recovered. The definition of each feature is given
below, where \(f(t)\) represents the height values of the first profile
at position \(t\) along the signature, and \(g(t)\) the height values of
the second. An indication of whether the feature is new since Hare,
Hofmann, and Carriquiry (2016) is also given:

\begin{itemize}
\tightlist
\item
  \textbf{ccf} (\%) is the maximum value of the Cross-Correlation
  function evaluated at the optimal alignment. The Cross-Correlation
  function is defined as
  \(C(\tau) = \int_{-\infty}^{\infty} f(t)g(t + \tau)dt\) where \(\tau\)
  represents the the lag of the second signature (T. Vorburger et al.
  2011).
\item
  \textbf{rough\_cor} (new) (\%) feature that quantifies the correlation
  between the two signatures after performing a second LOESS smoothing
  stage and then subtracting the result from the original signatures.
  This attempts to model the roughness of the surface after removing
  structure such as waviness.
\item
  \textbf{lag} (mm) Is the optimal lag for the ccf value.
\item
  \textbf{D} (mm) is the Euclidean vertical distance between each height
  value of the aligned signatures. This is defined as
  \(D^2 = \frac{1}{\text{\#}t}\sum_t \left[f(t) - g(t)\right]^2\). This
  is a measure of the total variation between two functions (Clarkson
  and Adams 1933).
\item
  \textbf{sd\_D} (mm) provides the standard deviation of the values of
  \emph{D} from above.
\item
  \textbf{signature\_length} (mm) is the overall length of the smallest
  of the two aligned signatures.
\item
  \textbf{overlap} (new) (\%) provides the percentage of the two
  signatures that overlap after the alignment stage.
\item
  \textbf{matches} (per mm) is the number of matching peaks/valleys
  (striae) per millimeter of the overlapping portion of the aligned
  signatures.
\item
  \textbf{mismatches} (per mm) is the number of mismatching
  peaks/valleys (striae) per millimeter of the overlapping portion of
  the aligned signatures.
\item
  \textbf{cms} (per mm) is the number of consecutively matching
  peaks/valleys (striae) per millimeter of the overlapping portion of
  the aligned signatures (Biasotti 1959, Wei Chu et al. (2013)).
\item
  \textbf{non\_cms} (per mm) is the number of consecutive mismatching
  peaks/valleys (striae) per millimeter of the overlapping portion of
  the aligned signatures.
\item
  \textbf{sum\_peaks} (per mm) is the the sum of the average heights of
  matched striae.
\end{itemize}

The features that are expressed on a per millimeter level are intended
to support the degraded land case, as discussed earlier. Note that the
computation differs slightly depending on the feature. For example, to
standardize the number of matches, the first count the raw number of
matching striae, and then divide this number by the length of the
overlapping region of the two lands (\texttt{overlap} from above). In
most cases, the overlapping region will be very close to the length of
the smaller signature. But depending on the alignment, this may not
always be true. This ensures that we do not punish a particular
cross-comparison for having a smaller region in which matches could
occur. On the other hand, the number of mismatches is divided by the
total length of the two aligned signatures, since mismatched striae can
occur even in the non-overlapping region of the two signatures.

The \texttt{rough\_cor} or Roughness Correlation is derived by
performing a second smoothing step, and subtracting the result from the
original signatures. This creates a new signature which eliminates some
of the overall structure, allowing global deformations to have less of
an influence on the model output. Where the roughness correlation is
most useful is in a scenario like Figure \ref{fig:roughcorgood}. This
figure shows the alignment of profile 40977 with 47600. The top panel
shows the smoothed signatures. The middle panel overlays a LOESS fit to
the average of the two signatures. Finally, to derive the roughness
correlation, this LOESS is subtracted from the original signature to
create a new set of roughness residuals, which are then given in the
bottom panel. Note that these two profiles do not match, yet the ccf is
0.7724. The roughness correlation (-0.0324) correctly indicates the lack
of matching. The roughness correlation acts as a check against false
positives which can arise when there are significant deformations in the
overall structure, as in the case with both these profiles.

\begin{figure}[htbp]
\centering
\includegraphics{degraded-paper_files/figure-latex/roughcorgood-1.pdf}
\caption{\label{fig:roughcorgood}Alignment of profile 40977 with 47600. The
top panel shows the smoothed signatures. The middle panel overlays a
LOESS fit to the average of the two signatures. Finally, to derive the
roughness correlation, this LOESS is subtracted from the original
signature to create a new set of roughness residuals, which are then
given in the bottom panel. Note that these two profiles do not match,
yet the ccf is 0.7724. The roughness correlation (-0.0324) correctly
indicates the lack of matching.}
\end{figure}

In a typical comparison between two profiles, such as in Figure
\ref{fig:roughcorfine}, the roughness correlation does not meaningfully
impact the matching probability given the presence of the ccf in the
model. In this figure, we see the alignment of profile 8752 with profile
136676. In this case, the waviness or the deformation pattern in the
signatures is less pronounced, and hence the resulting roughness
signature resembles the original signature more closely. These profiles
match, and both ccf (0.6891) and rough\_cor (0.7980) provide values
indicative of matching.

\begin{figure}[htbp]
\centering
\includegraphics{degraded-paper_files/figure-latex/roughcorfine-1.pdf}
\caption{\label{fig:roughcorfine}Alignment of profile 8752 with profile
136676. In this case, the waviness or the deformation pattern in the
signatures is more minor, and hence the resulting roughness signature
resembles the original signature more closely. These profiles match, and
both ccf (0.6891) and rough\_cor (0.7980) provide values indicative of
matching.}
\end{figure}

We can observe the distributions of both CCF and the Roughness
Correlation side by side, differentiating between known matches and
known non-matches. Figure \ref{fig:roughcordist} displays this as an
empirical CDF plot. It can be seen that the separation of known matches
and known non-matches along both the CCF and the Roughness Correlation
is quite strong and follows similar distributions (the known non-matches
are relatively symmetric, while the known matches are very skewed left).
However, some known non-matches with CCF values that would typically be
indicative of a match have relatively lower values for the Roughness
Correlation, which indicates that this feature could provide some added
value when it comes to discriminating between matches and non-matches.

\begin{figure}[htbp]
\centering
\includegraphics{degraded-paper_files/figure-latex/roughcordist-1.pdf}
\caption{\label{fig:roughcordist}Empirical CDFs of the Roughness Correlation
compared to the CCF for known matches and known non-matches. It can be
seen that the distribution of each feature for the known non-matches is
quite symmetric, while the distribution for each feature for the known
matches is skewed left.}
\end{figure}

\section{Model Training}\label{model-training}

Using these features, we can train a randomForest (Liaw and Wiener 2002)
model which attempts to predict whether two lands match given the value
of the features. There are currently three studies for bullets included
in the NIST Ballistics Toolmark Research Database. Those are Hamby (Set
252), Hamby (Set 44), and Cary. For purposes of the analysis we describe
in this paper, we exclude the Cary bullets from consideration, because
the study was designed to assess the persistence of striation markings
over a series of fires from the same barrel. Thus, every Cary bullet is
a known match to every other Cary bullet. Hence, we will consider Hamby
(Set 252) and Hamby (Set 44) only. This leaves us a total of 83,028
land-to-land comparisons, of which 1208 are among known matching land
impressions and 81,820 are among known non-matching land impressions.

We can now train the forest using the features we defined earlier. Using
the \texttt{caret} package (Jed Wing et al. 2016), we perform the
following partitioning scheme. Out of the 50 barrels total, ten knowns
and fifteen unknowns from each of the two Hamby sets, we hold out ten
barrels randomly as a testing set, and use the remaining 40 to train the
model. We repeat this procedure ten times and average the confusion
matrix in order to assess the model accuracy with different holdout
samples. Table \ref{tab:avgforest} displays the results in the form of a
confusion matrix on the test set, averaged over these ten independent
random forests trained on ten random barrel subsets. It can be seen that
false positives are exceedingly rare, but false negatives occur more
frequently (approximately 21` false negative land to land comparisons on
the test set, compared with an average of less than two false
positives).

\begin{table}[H]
\centering
\begin{tabular}{lr}
  \hline
Result & Count \\ 
  \hline
False Negative & 20.7 \\ 
  False Positive & 1.9 \\ 
  True Negative & 3716.8 \\ 
  True Positive & 56.5 \\ 
   \hline
\end{tabular}
\caption{The average confusion matrix for the 10 random forests. It can be seen that false positives are exceedingly rare, but false negatives occur more frequently.} 
\label{tab:avgforest}
\end{table}

These results suggest that our algorithm is too conservative in
predicting a match when in fact the bullets were fired from the same gun
barrel. We can break down the confusion matrix by the study from which
each of the two land impressions originated. Table
\ref{tab:avgforeststudy} shows the average confusion matrix for the 10
random forests, broken down by study. It can be seen that Hamby252 to
Hamby252 comparisons exhibit the fewest errors, while Hamby44 to Hamby44
comparisons exhibit the most errors on average. This intuitively makes
some sense given the potential presence of scanner operator effects,
which we address further in this section.

\begin{table}[H]
\centering
\begin{tabular}{lllll}
  \hline
Study & False Negative & False Positive & True Negative & True Positive \\ 
  \hline
Hamby252\_Hamby252 & 0.39\% & 0.02\% & 97.56\% & 2.03\% \\ 
  Hamby252\_Hamby44 & 0.45\% & 0.05\% & 98.31\% & 1.19\% \\ 
  Hamby44\_Hamby44 & 0.89\% & 0.09\% & 97.68\% & 1.34\% \\ 
   \hline
\end{tabular}
\caption{The average confusion matrix for the 10 random forests, broken down by study. It can be seen that Hamby252 to Hamby252 comparisons exhibit the fewest errors, while Hamby44 to Hamby44 comparisons exhibit the most on average.} 
\label{tab:avgforeststudy}
\end{table}

\section{Feature Robustness}\label{feature-robustness}

Our goal is to assess the robustness of the previously defined features
as it pertains to our bullet matching routines. This goal is both a
backward looking assessment of our previous results for full
land-to-land comparisons, and a forward looking one to help support the
case that these can be used in the degraded land case. As a first stage
to assessing this robustness, we produce parallel coordinate plots of
the various features based on true positive, true negative, false
positive, and false negative land-to-land matches. Figure \ref{fig:pcp}
displays these plots. The means of the true positive and the true
negative groups are shown as thick blue lines, respectively, in the two
panels. The dashed lines represent individual land to land comparisons,
with errors highlighted larger in red. It can be seen that the few false
positives tendt o have anomalously high \texttt{ccf} or
\texttt{matches}, while the false negatives tend have a lot of
variability, though tending to also have a high \texttt{ccf} value.

\begin{figure}[htbp]
\centering
\includegraphics{degraded-paper_files/figure-latex/pcp-1.pdf}
\caption{\label{fig:pcp}Parallel coordinate plot of the features based on
the random forest confusion matrix for true and false positives (above),
and true and false negatives (below). False positives tend to have some
feature anomalously high, while false negatives exhibit quite a spread,
sometimes having very large values of CCF or CMS, for example.}
\end{figure}

\subsection{Operator Effects}\label{operator-effects}

We attempt to quantify the effect of the study on the matching
probability by fitting a new random forest which is designed to predict
the study from which the scans came from based on the derived features.
It should be noted that the two sets of scans were performed by two
trained professionals at NIST, and therefore this analysis is intended
to shed light on how slightly different operating procedures for the
scans may lead to varying results in algorithms derived from these
scans. Ideally, if the assumption of independence between lands holds
across different operators, this forest should have poor performance -
The set of derived features should be relatively consistent among known
matches and known non-matches regardless of the study since the Hamby
data in both sets originated from the same gun barrels.

Table \ref{tab:studypred} shows the confusion matrix, with column
proportions, for the random forest with study as the response. It can be
seen that indeed the random forest performs poorly, as hoped, indicating
that a simple model to predict the study using the features available is
not enough to detect the operator effects.

\begin{table}[H]
\centering
\begin{tabular}{llll}
  \hline
Prediction $\backslash$ Actual & Hamby252\_Hamby252 & Hamby252\_Hamby44 & Hamby44\_Hamby44 \\ 
  \hline
Hamby252\_Hamby252 & 9.6\% & 8.32\% & 11.28\% \\ 
  Hamby252\_Hamby44 & 81.47\% & 81.06\% & 78.68\% \\ 
  Hamby44\_Hamby44 & 8.93\% & 10.63\% & 10.04\% \\ 
   \hline
\end{tabular}
\caption{Confusion Matrix (Column Proportions) for the random forest with study as the response. It can be seen that the random forest performs poorly, as hoped, indicating that a simple model to predict the study using the features available is not enough to detect the operator effects.} 
\label{tab:studypred}
\end{table}

Figure \ref{fig:ccfstudy} shows the distributions of the land-to-land
features, faceted by whether the lands are known to be fired from the
same gun barrel, across different study to study comparisons. The
distributions among the known non-matches seem relatively consistent
across study based on visual inspection. On the other hand, among known
matches, Hamby252 to Hamby252 comparisons exhibit more pronounced
features, including a higher average ccf, higher number of matches, and
higher value of sum\_peaks.

\begin{figure}[htbp]
\centering
\includegraphics{degraded-paper_files/figure-latex/ccfstudy-1.pdf}
\caption{\label{fig:ccfstudy}Distribution of the features, facetted by
match, for different study to study comparisons of lands.}
\end{figure}

Though visual inspection clearly shows differences, we can more formally
assess the differences between distributions with a Kolmogrov-Smirnov
test. Table \ref{tab:kstests} gives the results of pairwise tests, for
each feature, between different set comparisons, and between known
matches compared with known non-matches. Although most of the tests are
significant, looking at the raw values of the D statistic suggest that
the largest effect sizes do in fact occur in comparisons with two
Hamby252 lands, as the visual inspection of the boxplots also suggested.

\begin{table}[H]
\centering
\begin{tabular}{llllrlr}
  \hline
set1 & set2 & feature & matchtest & matchd & nonmatchtest & nonmatchd \\ 
  \hline
H252\_H252 & H252\_H44 & ccf & $<$ 0.0001 & 0.2723 & 1e-04 & 0.0189 \\ 
  H252\_H252 & H252\_H44 & cms & $<$ 0.0001 & 0.1751 & $<$ 0.0001 & 0.0245 \\ 
  H252\_H252 & H252\_H44 & D & $<$ 0.0001 & 0.2567 & $<$ 0.0001 & 0.1049 \\ 
  H252\_H252 & H252\_H44 & matches & $<$ 0.0001 & 0.1933 & $<$ 0.0001 & 0.0327 \\ 
  H252\_H252 & H252\_H44 & mismatches & $<$ 0.0001 & 0.2015 & 0.3537 & 0.0079 \\ 
  H252\_H252 & H252\_H44 & overlap & 0.0492 & 0.0984 & $<$ 0.0001 & 0.0276 \\ 
  H252\_H252 & H252\_H44 & rough\_cor & $<$ 0.0001 & 0.2647 & $<$ 0.0001 & 0.0970 \\ 
  H252\_H252 & H252\_H44 & sum\_peaks & 8e-04 & 0.1426 & 0.0015 & 0.0162 \\ 
  H252\_H252 & H44\_H44 & ccf & $<$ 0.0001 & 0.2160 & $<$ 0.0001 & 0.0257 \\ 
  H252\_H252 & H44\_H44 & cms & $<$ 0.0001 & 0.2515 & $<$ 0.0001 & 0.0467 \\ 
  H252\_H252 & H44\_H44 & D & $<$ 0.0001 & 0.2342 & $<$ 0.0001 & 0.1946 \\ 
  H252\_H252 & H44\_H44 & matches & $<$ 0.0001 & 0.2770 & $<$ 0.0001 & 0.0713 \\ 
  H252\_H252 & H44\_H44 & mismatches & $<$ 0.0001 & 0.2505 & 0.0414 & 0.0138 \\ 
  H252\_H252 & H44\_H44 & overlap & 0.2432 & 0.0906 & $<$ 0.0001 & 0.0408 \\ 
  H252\_H252 & H44\_H44 & rough\_cor & $<$ 0.0001 & 0.2242 & $<$ 0.0001 & 0.1718 \\ 
  H252\_H252 & H44\_H44 & sum\_peaks & 1e-04 & 0.1926 & $<$ 0.0001 & 0.0289 \\ 
  H252\_H44 & H44\_H44 & ccf & 0.1149 & 0.0883 & $<$ 0.0001 & 0.0259 \\ 
  H252\_H44 & H44\_H44 & cms & 0.111 & 0.0888 & $<$ 0.0001 & 0.0262 \\ 
  H252\_H44 & H44\_H44 & D & 0.2923 & 0.0724 & $<$ 0.0001 & 0.0906 \\ 
  H252\_H44 & H44\_H44 & matches & 0.0603 & 0.0977 & $<$ 0.0001 & 0.0423 \\ 
  H252\_H44 & H44\_H44 & mismatches & 0.3301 & 0.0700 & 0.1633 & 0.0096 \\ 
  H252\_H44 & H44\_H44 & overlap & 0.8231 & 0.0465 & 1e-04 & 0.0190 \\ 
  H252\_H44 & H44\_H44 & rough\_cor & 0.2671 & 0.0741 & $<$ 0.0001 & 0.0769 \\ 
  H252\_H44 & H44\_H44 & sum\_peaks & 0.047 & 0.1011 & 0.006 & 0.0147 \\ 
   \hline
\end{tabular}
\caption{Results for the Kolmogrov-Smirnov distributional test.} 
\label{tab:kstests}
\end{table}

These results strongly suggest the need for controlling for more effects
when performing the analysis. Specifically, microscope operator effects
resulting in variations in scan quality and scan parameters seem to play
a role in the utlimate performance of the matching algorithm. Land to
land comparisons from Hamby252 consistently result in more pronounced
expression of features among known matches, and therefore result in
higher accuracy in the random forest. Rigorous procedures to ensure scan
quality and consistency across operators need to be in place to minimize
the effect of the study and ensure that the assumption of land to land
independence is satisfied.

Another way to demostrate the study/operator effect is by observing the
distribution of our algorithm's ideal cross section by study. Figure
\ref{fig:crosscompare} gives the distributions of the ideal cross
sections by study. It can be seen that the Hamby44 ideal cross sections
are more likely to be close to the base of the bullet when compared to
the position of the ideal cross sections in Hamby252.

\begin{figure}[htbp]
\centering
\includegraphics{degraded-paper_files/figure-latex/crosscompare-1.pdf}
\caption{\label{fig:crosscompare}Distributions of the ideal cross sections
by study. It can be seen that the Hamby44 ideal cross sections are much
more likely to be close to the base of the bullet compared to Hamby252.}
\end{figure}

Indeed, another Kolmogorov-Smirnov test confirms a significant
difference in the distributions of these values
(\(D = 0.6239, p < 0.0001\)). This result strongly suggests that the
operator effect in the bullet scanning procedure must be taken into
account in order to assume pairwise independence of bullet land scans
between Hamby sets 252 and 44.

\subsection{Degraded Lands}\label{degraded-lands}

We now turn our attention to matching degraded bullet lands, in which
only fragments of the land can be recovered. Because the NIST database
currently contains only full bullet lands, we artificially degrade
bullets under some simplifying assumptions. Essentially, we delete
portions of lands to simulate the situation where we only recover a
fragment from the crime scene. We simulate various levels of degradation
from the left, right, and middle of the land impression. We vary the
proportion of the land impression that is recovered, between 100\% (no
degradation) and 25\% (significant degradation). For example, a
left-fixed 75\% scenario implies that the left hand portion of the land
was recovered, and the 25\% rightmost portion was lost. We will do this
by subsetting the signatures. Note that this is a a simplified scenario
because the signatures themselves are somewhat dependent on the data
that are missing because of the properties of the LOESS smoother.

Figure \ref{fig:senspe} gives the sensitivity (true positive rate) and
specificity (true negative rate) of the random forest predictions for
given levels of degradation. It can be seen that the sensitivity drops a
bit until 50\% of the land is available and then rises again. This
occurs because the algorithm begins producing more positive predictions
in general, likely as the result of the ccf being arbitrarily higher for
known non-matches due to the small signature. On the other hand, the
specificity drops dramatically for left, middle, and right fixed
degraded lands when less than 50\%. of the land impression is available
for examination. For a more in-depth exploration of the matching
probabilities, Figure \ref{fig:deghist} provides histograms of the
matching probability by degradation level and by known match versus
known non-match categories. The matching probabilities suffer when
compared with the probablities obtained from comparisons between full
land impressions in all cases. The jump seems to be most noticeable
beginning at about 25\% degradation (75\% land recovered), and the
algorithm struggles beyond 50\%.

\begin{figure}[htbp]
\centering
\includegraphics{degraded-paper_files/figure-latex/deghist-1.pdf}
\caption{\label{fig:deghist}Histograms of matching probability, facetted by
the degradation level and known match versus known non-match.}
\end{figure}

\begin{figure}[htbp]
\centering
\includegraphics{degraded-paper_files/figure-latex/senspe-1.pdf}
\caption{\label{fig:senspe}Sensitivity and specificity of the random forest
for given levels of degradation. It can be seen that both metrics
decline as a function of the land proportion, except for the
sensitivity, which rises for very low levels of the land proportion due
to an increase in the amount of positive predictions.}
\end{figure}

Figure \ref{fig:featexp} gives feature expression for known matches, as
a function of the proportion of land impression recovered. It is
immediately obvious that the variability in feature expression is large
when only a small fraction of the land is recovered, such as 25\%. For
instance, \texttt{sum\_peaks} and \texttt{cms} both drop, while
\texttt{D} rises. Interestingly, some of the features are better
expressed for the middle-fixed case. Overall, feature expression remains
relatively consistent as long as we recover 50\% or more of the land
impression. Feature \ref{fig:featexp2} shows the feature expression for
known non-matches by comparison. The non-matches don't exhibit the same
pattern of better feature expression for the middle-fixed case, except
perhaps for very low degradation levels. However, feature expression
rises as a function of the land proportion, which indicates why the
random forest begins predicting more positives, raising the sensitivity,
but drastically lowering the specificity.

\begin{figure}[htbp]
\centering
\includegraphics{degraded-paper_files/figure-latex/featexp-1.pdf}
\caption{\label{fig:featexp}Feature expression for known matches, as a
function of land proportion. It can be seen that when we fix the middle
portion of the bullet land, the features tend to be better expressed.}
\end{figure}

\begin{figure}[htbp]
\centering
\includegraphics{degraded-paper_files/figure-latex/featexp2-1.pdf}
\caption{\label{fig:featexp2}Feature expression for known non-matches, as a
function of land proportion. As expected by the fact that the false
positive rate increases as the land proportion decreases, so too does
the feature expression. However, unlike for the known matches, fixing
the middle portion of the land does not seem to lead to more expressed
features, except perhaps for very low degradation levels.}
\end{figure}

To come full circle, we now attempt to match a particular land which
exhibits bad tank rash. Figure \ref{fig:br924} provides an image of the
surface of this land impression. Due to the tank rash, this particular
land impression was originally excluded from consideration (see (Hare,
Hofmann, and Carriquiry 2016)) based on our subjective assessment of the
quality of the land. However, it appears that approximately half of the
bullet land remains relatively unaffected. We extract a signature from
the unaffected half and attempt to match this signature to its full
known match.

\begin{figure}[H]
\centering
\includegraphics[width=\linewidth]{images/br9-2-4-greyflip.png}
\caption{Land 4 of Bullet 2, from Barrel 9 of Hamby Set 252. It can be seen that this particular land exhibits some major tank rash on the right half.}
\label{fig:br924}
\end{figure}

Table \ref{tab:br924pred} shows the values of the features, after
extracting only the first 50\% of the Hamby Barrel 9 Bullet 2, 4th land
(and hence, simulating a left-fixed 50\% degraded scenario), compared
with a feature comparison between both full lands (and hence, including
the tank rash striae). The features are derived in a comparison with its
known match, the complete Bullet 1 third land fired from Barrel 9. The
features, including the ccf and the matches, are expressed enough to
(barely) indicate a match in the case of the degraded bullet. Using the
pre-trained random forest, the predicted matching probability is 52\%.
This is encouraging in that attempting to match the full bullet land, by
comparison, yields a matching probability of 0.0067\%. This is due to
the relatively higher values of the ccf, cms, and matches for the
degraded comparison, and suggests that the feature standardization is
working as intended.

\begin{table}[ht]
\centering
\begin{tabular}{lrr}
  \hline
Feature & Degraded Land & Full Land \\ 
  \hline
ccf & 0.6004 & 0.4442 \\ 
  rough\_cor & 0.3671 & 0.1633 \\ 
  D & 0.0018 & 0.0023 \\ 
  overlap & 0.9968 & 0.9968 \\ 
  matches & 10.2236 & 5.6275 \\ 
  mismatches & 7.5949 & 5.0713 \\ 
  cms & 9.2013 & 4.6043 \\ 
  non\_cms & 6.5823 & 2.5357 \\ 
  sum\_peaks & 12.0020 & 6.3148 \\ 
  matchprob & 0.5200 & 0.0067 \\ 
   \hline
\end{tabular}
\caption{Features extracted for a comparison of the full Hamby Barrel 9 Bullet 1 Land 3, with a left-fixed 50 percent degraded portion of Hamby Barrel 9 Bullet 2 Land 4. These two lands are known matches, and indeed the random forest does predict a match.} 
\label{tab:br924pred}
\end{table}

\addtocontents{toc}{\protect\newpage}

\section{From Lands to Bullets}\label{from-lands-to-bullets}

Another area that deserves more study is the question of generalizing
these algorithms to matching entire bullets rather than indvidual lands,
as would be done in a criminal justice application. One such approach is
to recognize that (at least for the Hamby bullets) there should be six
matching pairs of lands for any two bullets that were fired from the
same gun barrel. Therefore, for each pair of bullets, we can extract the
six highest matching probabilities and average them. If we do so, we
obtain a clear separation between the scores that are obtained when
matching bullets known to be matches and the scores obtained from known
non-matches. This is shown in Figure \ref{fig:firstscore}. No
known-matches have a score below 50\%, while all known non-matches have
a score below 10\%.

\begin{figure}[htbp]
\centering
\includegraphics{degraded-paper_files/figure-latex/firstscore-1.pdf}
\caption{\label{fig:firstscore}Score distributions for the naive approach to
bullet matching, for known matches and known non-matches.}
\end{figure}

We can improve on this approach by exploiting the rotation of the bullet
to compute a score. Under the assumption of land to land independence,
we can define the probability that two bullets match (M) as one minus
the probability that the two bullets do not match (NM). Exploiting the
idea that when two bullets do not match, none of the individual lands
match either, we can write the matching probability as the probability
that at least one land pair in the matrix matches. Specifically,

\begin{align*}
P(M) &= 1 - P(NM) \\
     &= 1 - (P(NM1) \times P(NM2) \times ... \times P(NM6)) \\
     &= 1 - ((1 - P(M1)) \times (1 - P(M2)) \times ... \times (1 - P(M6)))
\end{align*}

where \(M\) is the event that two bullets match, \(NM\) is the event
that two bullets do not match, \(M1\), \(M2\), \ldots{}, \(M6\) are the
probabilities of land one, land two, \ldots{}, land six matching, and
\(NM1\), \(NM2\), \ldots{}, \(NM6\) are the probabilities that land one,
land two, \ldots{}, land six do not match. However, to compute this
probability, we need to know the alignment of the two sets of lands.
Fortunately, the consistent rotation of the bullet permits this. For
instance, if we knew that land 1 of bullet 1 matches land 4 of bullet 2,
then we immediately know that land 2 of bullet 1 matches to land 5 of
bullet 2, land 3 of bullet 1 matches to land 6 of bullet 2, etc. Hence,
we can take look across six diagonals of the \(6 \otimes 6\) matrix
containing match probabilities. Table \ref{tab:diag} gives an example of
the matrix of matching probabilities between two sets of six lands from
bullets that are known matches. The matching diagonal is clear based on
the high probabilities (cell \((1, 3)\), cell \((2, 4)\), cell
\((3, 5)\), etc.) although it can be seen that one of the six
comparisons has a relatively lower matching probability. This procedure
is based on the Sequence Average Maximum (SAM) by Sensorfar (2017) in
their bullet matching software application \texttt{SensoMatch}. A
similar approach using the cross correlation maximum was first proposed
by W. Chu et al. (2010). Compared to that approach, ours uses random
forest based probabilities compared to correlation values allowing for
elements of probability theory to help determine the resulting bullet
match probability.

\begin{table}[ht]
\centering
\begin{tabular}{rrrrrrr}
  \hline
profile1\_id & 45604 & 46104 & 46601 & 47069 & 47600 & 48069 \\ 
  \hline
42594 & 0.0000 & 0.0000 & 1.0000 & 0.0000 & 0.0000 & 0.0000 \\ 
  43063 & 0.0000 & 0.0000 & 0.0000 & 1.0000 & 0.0067 & 0.0000 \\ 
  43581 & 0.0000 & 0.0000 & 0.0000 & 0.0000 & 0.8433 & 0.0000 \\ 
  44211 & 0.0000 & 0.0000 & 0.0133 & 0.0000 & 0.0000 & 0.6700 \\ 
  44568 & 1.0000 & 0.0000 & 0.0033 & 0.0000 & 0.0000 & 0.0000 \\ 
  45070 & 0.0000 & 1.0000 & 0.0000 & 0.0000 & 0.0000 & 0.0000 \\ 
   \hline
\end{tabular}
\caption{Matrix of matching probabilities between two sets of six lands from bullets that are known matches.} 
\label{tab:diag}
\end{table}

We now describe four methods for deriving a score from this matrix. The
results derived from these methods are shown in Figure
\ref{fig:allscores}. In \textbf{Method 1}, we derive a score by
computing the bullet matching probability on each set of six matrix
diagonals using the previously defined formula, under the assumption of
land to land indepndence. Finally, we take the maximum score obtained
out of the six results as the final matching score for a bullet pair.
After doing so, we can plot the scores for known matches and known
non-matches separately. It can be seen that the known matches all have
scores of around 100\%, while no non-match achieves a score of above
30\%, and hence this procedure provides perfect discrimination between
all pairs of bullets between and within the two Hamby datasets.

On the other hand, \textbf{Method 2} is obtained by flipping this
procedure around by assuming that a match occurs if and only if all six
lands match. As it turns out, this does not discriminate quite as well.
Every known bullet non-match achieves a score of about zero, but so do
about 15 known bullet matches. This method performs poorly because our
matching algorithm exhibits a larger false negative rate than the rate
of false positives. Multiplying the probabilities together compounds the
issue of false negatives and leads to some misidentification of matching
bullets.

\textbf{Method 3} is a hybrid of these two approaches, where we average
the probabilities along the diagonal rather than multiplying those
probabilities. Now, we once again differentiate the two groups well with
no known non-match achieving a score above 10\%, and no known match with
a score below 40\%.

One more approach to generate bullet matching scores, which we call
\textbf{Method 4}, would exploit the SAM procedure on individual
features. For each diagonal in the \(6 \otimes 6\) matrix, we can
compute an average value for each feature in our model. This yields six
sets of feature values for all six diagonals. We can then feed all six
sets of features into the random forest in order to obtain a matching
probability for each, taking the highest resulting probability to locate
the diagonal and thus identify land to land alignment. It can be seen
that while this procedure does discriminate well, it yields some false
negatives (matching bullets that our forest identifies as a non-match).

\begin{figure}[htbp]
\centering
\includegraphics{degraded-paper_files/figure-latex/allscores-1.pdf}
\caption{\label{fig:allscores}Distribution of matching scores using four
methods. Method 1 assumes a match if at least one pair of lands match.
Method 2 assumes a match if all pairs of lands match. Method 3 averages
the probabilities instead of multiplying them. Finally, Method 4 uses a
SAM procedure on the feature values for known matches compared to known
non-matches. While these methods have various levels of discriminatory
power, and rely on slightly different assumptions, they do show clear
and significant separation between matching bullets and non-matching
bullets in general.}
\end{figure}

\section{Conclusion}\label{conclusion}

In this paper, we have introduced a set of robust features that can be
used to train bullet matching models. We have used these features to
train a random forest and assess its out-of-sample accuracy. In doing
so, we noted strong evidence of operator effects that resulted in
differences in the quality of the microscope scans. These effects were
noted despite the experience of the individuals conducting the scans,
which implies that such effects could quite likely be more pronounced
when scans are done by those more inexperienced, or with fewer standard
operating procedures in place.

While these effects were clearly identified, the best approach to
account for them in practice is less clear. In the ideal case, bullets
fired from a particular gun barrel should yield surface scans that are
of identical quality and properties, regardless of the operator
performing the scan. To achieve this, rigorous standards may need to be
put in place with regards to the alignment of the bullet under the
objective, and the procedure used to scan the bullet surface. To
appropriately design a set of best practices requires more research. For
instance, because of the significant difference between the placement of
the ideal cross section across the two studies, a best practice may
specify the margin from the edge of the objective at which the bullet
can be placed.

We began exploring the robustness of the matching algorithm proposed in
(Hare, Hofmann, and Carriquiry 2016) to land degradation. As suspected,
the algorithm performance declines as a function of the rate of
degradation. However, there is a relatively clear threshold around about
50\%; if 50\% of the land or more is recovered, the algorithm still
performs reasonably well. When the proportion of the land that is
recovered is below 50\%, the accuracy with which we can compare land
impressions is low.

Finally, one pleasing conclusion from these results is the fact that
generalizing them to full bullet comparisons rather than the land to
land level appears to work quite well. Depending on the assumptions
made, the out of sample accuracy of bullet to bullet comparisons can
range from nearly perfect to perfect. This result is encouraging in that
real world use of these algorithms would be done on the bullet level,
assuming enough of the bullet was recovered to make these procedures
possible.

As we have stated before, the lack of 3D images of bullets available in
the public domain limits the extent to which these algorithms can be
tested and validated. The degraded land simulation itself may be too
simplistic and not faithfully represent realistic scenarios. However, as
more data are collected, we can continue to update, train, and test the
matching algorithm in order to improve its performance in real datasets.

\clearpage

\section*{References}\label{references}
\addcontentsline{toc}{section}{References}

\hypertarget{refs}{}
\hypertarget{ref-pcast2016}{}
Advisors on Science, President's Council of, and Technology. 2016.
``Report on Forensic Science in Criminal Courts: Ensuring Scientific
Validity of Feature-Comparison Methods.''
\url{https://www.whitehouse.gov/sites/default/files/microsites/ostp/PCAST/pcast_forensic_science_report_final.pdf}.

\hypertarget{ref-biasotti:1959}{}
Biasotti, Alfred A. 1959. ``A Statistical Study of the Individual
Characteristics of Fired Bullets.'' \emph{Journal of Forensic Sciences}
4 (1): 34--50.

\hypertarget{ref-chu:2011}{}
Chu, W., J. Song, T. Vorburger, R. Thompson, and R. Silver. 2011.
``Selecting Valid Correlation Areas for Automated Bullet Identification
System Based on Striation Detection.'' \emph{Journal of Research of the
National Institute of Standards and Technology} 116 (3): 649.

\hypertarget{ref-chu:2010}{}
Chu, W., J. Song, T. Vorburger, J. Yen, S. Ballou, and B. Bachrach.
2010. ``Pilot study of automated bullet signature identification based
on topography measurements and correlations.'' \emph{J. Forensic Sci.}
55 (2): 341--47.

\hypertarget{ref-thompson:2013}{}
Chu, Wei, Robert M Thompson, John Song, and Theodore V Vorburger. 2013.
``Automatic identification of bullet signatures based on consecutive
matching striae (CMS) criteria.'' \emph{Forensic Science International}
231 (1--3): 137--41.

\hypertarget{ref-clarkson1933definitions}{}
Clarkson, James A, and C Raymond Adams. 1933. ``On Definitions of
Bounded Variation for Functions of Two Variables.'' \emph{Transactions
of the American Mathematical Society} 35 (4). JSTOR: 824--54.

\hypertarget{ref-national2004forensic}{}
Council, National Research, and others. 2004. \emph{Forensic Analysis:
Weighing Bullet Lead Evidence}. National Academies Press.

\hypertarget{ref-giannelli:2011}{}
Giannelli, Paul C. 2011. ``Ballistics Evidence Under Fire.''
\emph{Criminal Justice} 25 (4): 50--51.

\hypertarget{ref-hamby:2009}{}
Hamby, James E., David J. Brundage, and James W. Thorpe. 2009. ``The
Identification of Bullets Fired from 10 Consecutively Rifled 9mm Ruger
Pistol Barrels: A Research Project Involving 507 Participants from 20
Countries.'' \emph{AFTE Journal} 41 (2): 99--110.

\hypertarget{ref-2016arXiv160105788H}{}
Hare, E., H. Hofmann, and A. Carriquiry. 2016. ``Automatic Matching of
Bullet Lands.'' \emph{ArXiv E-Prints}, January.

\hypertarget{ref-caretpkg}{}
Jed Wing, Max Kuhn. Contributions from, Steve Weston, Andre Williams,
Chris Keefer, Allan Engelhardt, Tony Cooper, Zachary Mayer, et al. 2016.
\emph{Caret: Classification and Regression Training}.
\url{https://CRAN.R-project.org/package=caret}.

\hypertarget{ref-randomForest}{}
Liaw, Andy, and Matthew Wiener. 2002. ``Classification and Regression by
RandomForest.'' \emph{R News} 2 (3): 18--22.
\url{http://CRAN.R-project.org/doc/Rnews/}.

\hypertarget{ref-NAS:2009}{}
National Research Council. 2009. \emph{Strengthening Forensic Science in
the United States: A Path Forward}. Washington, DC: The National
Academies Press.
doi:\href{https://doi.org/10.17226/12589}{10.17226/12589}.

\hypertarget{ref-petraco:2012}{}
Petraco, Nicholas, and Helen Chan. 2012. \emph{Application of Machine
Learning to Toolmarks: Statistically Based Methods for Impression
Pattern Comparisons}. Mannheim, Germany: Bibliographisches Institut AG.

\hypertarget{ref-riva:2014}{}
Riva, Fabiano, and Christophe Champod. 2014. ``Automatic Comparison and
Evaluation of Impressions Left by a Firearm on Fired Cartridge Cases.''
\emph{Journal of Forensic Sciences} 59 (3): 637--47.
doi:\href{https://doi.org/10.1111/1556-4029.12382}{10.1111/1556-4029.12382}.

\hypertarget{ref-sensorfar}{}
Sensorfar. 2017. \emph{SensoMATCH Bullet Comparison Software}.

\hypertarget{ref-vorburger:2011}{}
Vorburger, T.V., J.-F. Song, W. Chu, L. Ma, S.H. Bui, A. Zheng, and T.B.
Renegar. 2011. ``Applications of Cross-Correlation Functions.''
\emph{Wear} 271 (3--4): 529--33.
doi:\href{https://doi.org/http://dx.doi.org/10.1016/j.wear.2010.03.030}{http://dx.doi.org/10.1016/j.wear.2010.03.030}.


\end{document}
